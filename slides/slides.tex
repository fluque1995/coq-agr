\documentclass[10pt]{beamer}

\usetheme[progressbar=frametitle]{metropolis}
\usepackage{appendixnumberbeamer}

\usepackage{booktabs}
\usepackage[scale=2]{ccicons}

\usepackage{pgfplots}
\usepgfplotslibrary{dateplot}

\usepackage[utf8]{inputenc}
\usepackage[spanish]{babel}

\usepackage{xspace}
\newcommand{\themename}{\textbf{\textsc{metropolis}}\xspace}

\title{Demostradores interactivos de teoremas}
\subtitle{Introducción a Coq}
\date{\today}
\author{Francisco Luque \\ Ignacio Mas}
\institute{Universidad de Granada}

\begin{document}

\maketitle

\begin{frame}{Contenidos}
  \setbeamertemplate{section in toc}[sections numbered]
  \tableofcontents[hideallsubsections]
\end{frame}

\section{Introducción}

\begin{frame}{Demostradores interactivos de teoremas}

  \begin{itemize}
  \item Herramientas orientadas a la demostración de teoremas por ordenador
  \item Colaboran con un usuario humano para comprobar la corrección de las demostraciones
  \item Basadas en el isomorfismo de Curry-Howard
  \item Agda, Isabelle, Coq
  \end{itemize}

\end{frame}

\section{Historia}

\begin{frame}{Historia}

  \begin{itemize}
  \item
  \end{itemize}

\end{frame}

\begin{frame}{Historia}

  \begin{itemize}
  \item A mediados del siglo XX comienza a pensarse que los ordenadores pueden
    usarse como herramienta para hacer demostraciones
  \item Comienzan a aparecer herramientas que permiten comprobar la corrección
    de pruebas matemáticas por ordenador
  \end{itemize}

\end{frame}

\section{Fundamentación teórica}

\begin{frame}[fragile]{Lógica intuicionista}

  \begin{itemize}
  \item Desarrollada por Arend Heyting en el marco del intuicionismo de Brouwer
  \item Debilita la lógica clásica eliminando el axioma del tercio excluso, i.e., $a \lor \neg a$
  \end{itemize}

\end{frame}

\begin{frame}[fragile]{Isomorfismo de Curry-Howard}

  \begin{itemize}
  \item Correspondencia unívoca entre el universo de las proposiciones y el de los tipos
  \item Las proposiciones se corresponden con tipos; las demostraciones de los teoremas, con los programas que computan las funciones
  \item La proposiciones serán ciertas \textit{si, y sólo si}, el tipo correspondiente está \textbf{habitado}
  \end{itemize}

\end{frame}

\begin{frame}[fragile]{Isomorfismo de Curry-Howard (II)}

  \begin{itemize}
  \item $a \land b \iff (A, B)$
  \item $a \lor b \iff A + B$
  \item $a \implies b \iff A \leftarrow B$
  \item $\bot \iff Void$
  \item $\neg a \iff A \leftarrow Void$
  \end{itemize}

\end{frame}

\section{CoQQQ}

\begin{frame}[fragile]{Coq}

  \begin{itemize}
  \item Programilla de abyecta sintaxis que lo peta para demostrar teoremillas
  \item Nuestro quebradero de cabeza de las últimas dos semanas
  \end{itemize}

\end{frame}

\section{Pruebas sobre álgebras usando Coq}

\begin{frame}[fragile]{Pruebas}

  \begin{itemize}
  \item Aquí viene la magia
  \end{itemize}

\end{frame}

\end{document}
