\documentclass[10pt]{beamer}

\usetheme[progressbar=frametitle]{metropolis}
\usepackage{appendixnumberbeamer}

\usepackage{booktabs}
\usepackage[scale=2]{ccicons}

\usepackage{pgfplots}
\usepgfplotslibrary{dateplot}

\usepackage[utf8]{inputenc}
\usepackage[spanish]{babel}

\usepackage{xspace}
\newcommand{\themename}{\textbf{\textsc{metropolis}}\xspace}

\title{Demostradores interactivos de teoremas}
\subtitle{Introducción a Coq}
\date{\today}
\author{Francisco Luque \\ Ignacio Mas}
\institute{Universidad de Granada}

\begin{document}

\maketitle

\begin{frame}{Contenidos}
  \setbeamertemplate{section in toc}[sections numbered]
  \tableofcontents[hideallsubsections]
\end{frame}

\section{Introducción}

\begin{frame}{Demostradores interactivos de teoremas}

  \begin{itemize}
  \item Herramientas orientadas a la demostración de teoremas por
    ordenador
  \item Colaboran con un usuario humano para comprobar la corrección
    de las demostraciones
  \item Basadas en el isomorfismo de Curry-Howard
  \item Agda, Isabelle, Coq
  \end{itemize}

\end{frame}

\section{Historia}

\begin{frame}[fragile]{Historia}

  \begin{itemize}
  \item Desde principios del siglo XIX, se intenta construir máquinas
    que automaticen cálculos (máquina de calcular y máquina analítica
    de Charles Babbage), basadas en sistemas mecánicos
  \item A principios de los años 30 del siglo XIX, Alonzo Church
    desarrolla el cálculo lambda, que consiste en un sistema formal
    que especifica las funciones que pueden ser calculables por un
    ordenador
  \item En 1936, Alan Turing describe en su artículo \textit{On
      Computable Numbers, with an Application to the
      Entscheidungsproblem} los límites de la computación, demostrando
    que no todos los problemas pueden ser resueltos utilizando un
    ordenador, con una aproximación distinta aunque equivalente a la
    que había dado Church
  \end{itemize}

\end{frame}

\begin{frame}{Historia}

  \begin{itemize}
  \item A mediados del siglo XX comienza a pensarse que los
    ordenadores pueden usarse como herramienta para hacer
    demostraciones
  \item Comienzan a aparecer herramientas que permiten comprobar la
    corrección de pruebas matemáticas por ordenador
  \item En 1954, Martin Davis consigue demostrar utilizando una
    máquina basada en tubos de vacío que la suma de dos números pares
    es par
  \end{itemize}

\end{frame}

\begin{frame}{Asistentes de demostración y demostradores de teoremas}

  \begin{itemize}
  \item Los demostradores de teoremas son sistemas orientados a la
    demostración automática, sin la cooperación de un usuario humano
  \item Los asistentes de demostración son sistemas orientados a
    comprobar la corrección de una demostración dada por un humano
  \item Los segundos son más simples, pero también menos potentes
  \item Nos centraremos en los segundos
  \end{itemize}

\end{frame}

\section{Fundamentación teórica}

\begin{frame}[fragile]{Lógica intuicionista}

  \begin{itemize}
  \item Desarrollada por Arend Heyting en el marco del intuicionismo
    de Brouwer
  \item Debilita la lógica clásica eliminando el axioma del tercio
    excluso, i.e., $a \lor \neg a$
  \end{itemize}

\end{frame}

\begin{frame}[fragile]{Isomorfismo de Curry-Howard}

  \begin{itemize}
  \item Correspondencia unívoca entre el universo de las proposiciones
    y el de los tipos
  \item Las proposiciones se corresponden con tipos; las
    demostraciones de los teoremas, con los programas que computan las
    funciones
  \item La proposiciones serán ciertas \textit{si, y sólo si}, el tipo
    correspondiente está \textbf{habitado}
  \end{itemize}

\end{frame}

\begin{frame}[fragile]{Isomorfismo de Curry-Howard (II)}

  \begin{itemize}
  \item $a \land b \iff (A, B)$
  \item $a \lor b \iff A + B$
  \item $a \implies b \iff A \leftarrow B$
  \item $\bot \iff Void$
  \item $\neg a \iff A \leftarrow Void$
  \end{itemize}

\end{frame}

\section{Coq}

\begin{frame}[fragile]{Coq}

  \begin{itemize}
  \item Demostrador interactivo de teoremas
  \item Comienza su desarrollo en 1984 en el instituto INRIA
    (Rocquencourt, cerca de París)
  \item Escrito en OCaml, que es la versión orientada a objetos de
    Caml (\textit{Categorical Abstract Machine Language}), ambos
    desarrollados en primera instancia en la \textit{École Normale
      Superieure} de París y posteriormente en el INRIA
  \item A pesar de ser un asistente de comprobación de teoremas,
    incorpora algunas directivas propias de los sistemas automáticos
    de demostración
  \item Se utilizó en la demostración del Teorema de los Cuatro Colores,
    cuya demostración completa se dió en 2004
  \end{itemize}

\end{frame}

\section{Pruebas sobre álgebras usando Coq}

\begin{frame}[fragile]{Pruebas}

  \begin{itemize}
  \item Aquí viene la magia
  \end{itemize}

\end{frame}

\end{document}
