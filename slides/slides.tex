\documentclass[10pt]{beamer}

\usetheme[progressbar=frametitle]{metropolis}
\usepackage{appendixnumberbeamer}

\usepackage{booktabs}
\usepackage[scale=2]{ccicons}

\usepackage{pgfplots}
\usepgfplotslibrary{dateplot}

\usepackage[utf8]{inputenc}
\usepackage[spanish]{babel}

\usepackage{xspace}
\newcommand{\themename}{\textbf{\textsc{metropolis}}\xspace}

\title{Demostradores interactivos de teoremas}
\subtitle{Introducción a Coq}
\date{\today}
\author{Francisco Luque \\ Ignacio Mas}
\institute{Universidad de Granada}

\begin{document}

\maketitle

\begin{frame}{Contenidos}
  \setbeamertemplate{section in toc}[sections numbered]
  \tableofcontents[hideallsubsections]
\end{frame}

\section{Introducción}

\begin{frame}[fragile]{Demostradores de teoremas}

  \begin{itemize}
  \item Herramientas orientadas a la demostración de teoremas por ordenador
  \item Basadas en el isomorfismo de Curry-Howard
  \end{itemize}

\end{frame}

\section{Historia}

\begin{frame}[fragile]{Historia}

  \begin{itemize}
  \item A mediados del siglo XX comienza a pensarse que los ordenadores pueden
    usarse como herramienta para hacer demostraciones
  \item Comienzan a aparecer herramientas que permiten comprobar la corrección
    de pruebas matemáticas por ordenador (
  \end{itemize}

\end{frame}

\section{Isomorfismo de Curry-Howard}

\begin{frame}[fragile]{Isomorfismo de Curry-Howard}

  \begin{itemize}
  \item Wea abyecta que justifica que esto se pueda usar
  \end{itemize}

\end{frame}

\section{CoQQQ}

\begin{frame}[fragile]{Coq}

  \begin{itemize}
  \item Programilla de abyecta sintaxis que lo peta para demostrar teoremillas
  \item Nuestro quebradero de cabeza de las últimas dos semanas
  \end{itemize}

\end{frame}

\section{Pruebas sobre álgebras usando Coq}

\begin{frame}[fragile]{Pruebas}

  \begin{itemize}
  \item Aquí viene la magia
  \end{itemize}

\end{frame}

\end{document}
